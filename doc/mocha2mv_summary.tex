\documentclass[a4paper,12pt]{article}
\title{Mocha2MV - Using reactive modules with VIS}
\author{Sebastian Gfeller\\
Supervisor: Barbara Jobstmann}
\begin{document}

\maketitle

\section{Introduction}
Mocha2MV provides a utility that translates a system
specification in the Reactive Modules format (as accepted by jMocha) into
BLIF-MV, a format accepted by VIS. Furthermore, an extension package allows to
directly use Reactive Modules with VIS.

\section{Architecture}
The main conversion code was written in Java, reusing the jMocha parse
tree (this is the only part needed, so if jMocha does not run on your
platform, you might still get lucky). The VIS bindings are written in
C code and are simply calling the conversion JAR, passing the output
to VIS-internal commands.

\subsection{Translation Details}
BLIF-MV is a fairly low-level format based on models which can contain
subcircuits, latches and input-output tables. Reactive Modules are
composed of modules containing atoms that control variables. Both in
Reactive Modules and in BLIF-MV, variable updates may be nondeterministic,
but VIS only supports this in the form of additional pseudo-input variables.

Here is a simple Mocha module that describes a traffic light:

\begin{verbatim}
type color is { green, orange, red }

module minitraffic is
       interface light: color;

       lazy atom CtrlLight controls light reads light
       	    init
		[] true -> light' := green;
	    update
		[] light = green -> light' := orange;
		[] light = orange -> light' := red;
		[] light = red -> light' := green;
\end{verbatim}

This is translated to the following BLIF-MV code:

\begin{verbatim}
.model minitraffic
# Non-deterministic pseudo-inputs
.inputs nd0 
.outputs light light'
.mv light,light' 3 green orange red
.mv nd0 2
.latch light' light
# CtrlLight
.reset light
green 
.table nd0 light  -> light' 
0 green orange 
0 orange red 
0 red green 
1 green =light 
1 orange =light 
1 red =light 
# Filling in the unused nondet combinations.
.end
\end{verbatim}

Variables are represented by latches and the laziness of the atom is
translated in a second update choice where the value of the latch stays
the same. Module composition is simulated by subcircuits.

\subsection{The Translation implementation}
The translation is done in multiple steps. In a first step, all types
are collected. Afterwards, the models are specified on the basis of the
input modules. Finally, the input-output tables are filled by evaluating
the guarded expressions and assignments on the set of possible inputs.

While VIS allows for a more compact form, expressing multiple inputs on
one line, this does not correspond well to the arithmetic expressivity
of reactive modules, so no size optimization is performed. This means
that tables can get quite large when using big range types. Usually,
this problem can be avoided by letting atoms only read and control few
variables.

\section{Installation}
The Mocha2MV distribution consists of two parts, a JAR and
command line tool to translate a reactive modules specification
into a BLIF-MV file and bindings for VIS.

\subsection{Installing the command line tool}
To build the command line tool and corresponding JAR file for
a Unix system, type

\begin{verbatim}
% ./install.sh
\end{verbatim}

In the main directory. This launches the ANT task to compile the
jar and places an executable (that points to this exact JAR) in
/usr/local/bin.

There are no executables provided for Windows, but you can nevertheless
use the jar directly to convert reactive modules. For this, just launch
the default ant task.

\subsection{VIS bindings}
The VIS bindings are a bit trickier to install. First, you'll have
to check whether you can build VIS on your system. Let \$VIS denote
the VIS source directory and \$MOCHA2MV the Mocha2MV source directory.

\begin{enumerate}
\item Copy the directory ``mocha'' to VIS:

\begin{verbatim}
% cp -R $MOCHA2MV/mocha $VIS/src/mocha
\end{verbatim}

\item Let VIS know of the ``read\_mocha'' command: For this, open
\$VIS/src/vm/vm.h and add the following include in the section ``nested
includes'' (line 50)

\begin{verbatim}
#include "mocha.h"
\end{verbatim}

\item Launch the command initialization at VIS startup: in the file
\$VIS/src/vm/vmInit.c , add the following at the end of the function
VmInit:

\begin{verbatim}
Mocha_Init();
\end{verbatim}

\item Add mocha to the list of packages to build in \$VIS/Makefile, line 98:

\begin{verbatim}
ALL_PKGS = ... var vm mocha
\end{verbatim}

\item Finally, make; make install in \$VIS to install the
version with VIS.
\end{enumerate}

\section{Usage}
After the complicated installation, the conversion can be invoked pretty
easily. To translate and load the file 3bitcounter.rm, you would type inside
VIS

\begin{verbatim}
vis> read_mocha ./3bitcounter.rm
vis> init_verify
\end{verbatim}

It will sometimes occur that Mocha2MV fails to translate the input file
correctly. Most of the time, this will fail during the table generation
phase. The error thrown should give some indication what construct is
missing in the translator. In the following section, some limitations
are named. However, it might also be possible, that an incomplete model is
generated. In this case, the init\_verify command will complain about it.

\section{Limitations}
Mocha2MV does not yet translate all valid Reactive Module specifications. The
following things are missing:

\begin{description}
\item[Array types and Bitvectors] Array types and bitvectors are not supported at the moment. They could,
however easily be represented by multiple latches. Additions to the Expression
evaluator would also be needed.
\item[Compact sequence representations] In the same fashion, sequence types are used
in their extended form and not in a binary representation, which would be
much smaller.
\item[Event atoms] While event atoms are translated, they don't produce usable output if more than one variable is awaited.
\end{description}
\end{document}
